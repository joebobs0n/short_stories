\documentclass[11pt]{memoir}
\usepackage{charter}

\usepackage[utf8]{inputenc}
\usepackage{graphicx}
\usepackage{xcolor}
\usepackage{setspace}
\usepackage[normalem]{ulem}

\title{The Old Woman}
\author{Andy Monk}

\begin{document}

\clearpage\maketitle
\thispagestyle{empty}


\frontmatter
% \tableofcontents

\chapter{Prelude}

    If you were to take the worn, east-ward road out of Langton to its end and then continue down the game trail, the week's journey would bring you to Gransford. Seamlessly coddled by forest on one side and ensconsed by cliffs on the other, the time forlorn community rests, refusing to be snuffed from reality. It's very existence secluded or ignored by the world beyond exemplified by its absence of visitors and the like. Depending on whom you were to ask, this was a prison or a palace; the trees upon the one edge being bars or castle walls, the cliffs either a moat or void.

    Many of the trees skirting the edge of the forest were younger and  provided a place for the children to play as, despite being a thick wood, there were no large animals or dangers to be had within. The kids would imagine the pines to be lost friends of times long past, and ocasionally at the setting sun, the last rays of daylight would flick across the new trees, puppeteering the shadows into dances until their finale at the sun's ultimate descent. The kids found particular joy in playing amongst their shadowy kinsfolk, waltzing until their parents drew them in for the night.

    When the children grew, they would bid farewell to their coniferous playmates, dawn Medallions of Plenty, and find their way to the fishing cliffs. There they would find perch upon the worn ground, dangle their eager legs, and dream the day away, fishing with their parents. Their dreams would grow, just as their bodies, from the frivolities of tag and hide-and-seek to marriage and bringing forth children of their own. Little was as unmovable as the sediment of the shorn coastline save the cycle of life in this quiet, forgotten corner of the world.

    Life in Gransford is quaint, quiet, and immutable. All was, is, and will continue to be the same and each resident held the same fate. Oft I would contemplate the simplicity of this life as I'd nestle upon my spot, grazing the depths below with my fishing rod. Times like this were precious as it reminded me of my younger years when my mother was still around and taught me.

    I had always looked up to her, as did the community, since she was a shining pillar of faith and purity. A small woman of five feet three inches and petite frame, she loved our traditions and what this place meant. She would often get swept away in prose of the beauty and correctness of our blessed lives here. In contrast, when preaching about the Outside, she would speak as one who had been there, but that was not so. She knew that the correctness of our ways here meant that those who dwelt in their gauche, gaudy ways could not begin to hope for communion with the Gods at their life's end, save they recentered their morals. Even the tussles between the Jeswalops and Grishams seemed little more than comedic routines when she was around.

    As her sole child and son, I like to believe that I held a special place in her heart. Her smile was particularly warm towards me. We even had our special place of communion at the lone white ash tree, a small hike north of town upon the cliffs. There she would run her delicate fingers through my wiry locks and hum sweet lullabies. Wasting away the hours, watching the lazy horizon, she would give private lessons and indulge my curiosities. These lessons were not sanctioned by Lady Yağmur, but they did not wane from those given by the lady herself; but like the special spices of homecooked meals, my mother's opinions and perspective would enhance the instruction and bring greater meaning and conviction.

    To have that conviction put into scruitiny by one innocent statement, oh how weak my heart must be. To have the pearls of youth snatched so effortlessly. Such is my plight.

\mainmatter



\backmatter


\end{document}
